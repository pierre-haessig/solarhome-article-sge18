\documentclass[a4paper,10pt,twocolumn]{article}

\usepackage[utf8x]{inputenc}
\usepackage[french]{babel}
\usepackage{graphicx,times}
\usepackage[T1]{fontenc}
\usepackage{amsmath}
\usepackage[font=footnotesize]{caption}
\usepackage{fancyhdr}
\usepackage[explicit]{titlesec}
\usepackage{hyperref}
%\usepackage[square,comma,numbers]{natbib}
\usepackage{tabularx}

\newcolumntype{L}[1]{>{\raggedright\arraybackslash}p{#1}}
\newcolumntype{C}[1]{>{\centering\arraybackslash}p{#1}}
\newcolumntype{R}[1]{>{\raggedleft\arraybackslash}p{#1}}

%\renewcommand{\bibsection}{}
%\def\bibfont{\footnotesize}
%\setlength{\bibsep}{0.2em}
%\setlength{\bibhang}{10em}

\hypersetup{colorlinks=true, urlcolor=blue, urlbordercolor={0 0 1}, citecolor=black, citebordercolor={1 1 1}}

\addto\captionsfrench{\def\figurename{Fig.}}
\addto\captionsfrench{\def\tablename{Tableau}}

\captionsetup[figure]{labelsep=period, justification=raggedright, singlelinecheck=false}
\captionsetup[table]{labelsep=period, justification=centering, singlelinecheck=false}

\parindent 10pt

\setlength{\voffset}{-1.3in}
\setlength{\topmargin}{1.25cm}
\setlength{\headheight}{1.125cm}
\setlength{\headsep}{0cm}

\setlength{\hoffset}{-1in}
\setlength{\oddsidemargin}{1.3cm}
\setlength{\evensidemargin}{1.3cm}

\setlength{\textheight}{25cm}%23.5
\setlength{\textwidth}{18.5cm}

\setlength{\headsep}{0.67cm}
\setlength{\columnwidth}{8.75cm}
\setlength{\columnsep}{0.63cm}

\setlength{\abovecaptionskip}{0em}
\setlength{\belowcaptionskip}{0em}

\titleformat{\section}
  {\normalfont}{\thesection.}{0.5em}{\MakeUppercase{#1}}
\titleformat{\subsection}
  {\normalfont\itshape}{\thesubsection.}{1.5em}{#1}
\titleformat{\subsubsection}
  {\normalfont\itshape}{\thesubsubsection.}{1.5em}{#1}
	
\titlespacing\section{0pt}{1em}{0.5em}
\titlespacing\subsection{0pt}{1em}{0.5em}
\titlespacing\subsubsection{0pt}{1em}{0.5em}

\fancyhf{}
\fancyhead[R]{\fontsize{8pt}{8pt}\selectfont \textbf{S}YMPOSIUM DE \textbf{G}ENIE \textbf{E}LECTRIQUE (SGE 2018), 3-5 JUILLET 2018, NANCY, FRANCE}
\renewcommand{\headrulewidth}{0pt}


\pagestyle{empty}


\title{
\fontsize{24pt}{24pt}\selectfont
Gestion d'énergie avec entrées incertaines : \\
quel algorithme choisir ?\\
Illustration pour un système PV-stockage.
}

\newcommand\tsp[1]{\textsuperscript{#1}}

\author{
\fontsize{11pt}{11pt}\selectfont
Pierre HAESSIG\tsp{*}\\
\fontsize{10pt}{10pt}\selectfont
\tsp{*}CentraleSupélec -- IETR
}

\date{}


\begin{document}

\maketitle
\thispagestyle{fancy}


\fontsize{9pt}{9pt}\selectfont
\textbf{RÉSUME --
Le pilotage optimal des systèmes énergétiques nécessite l'emploi d'algorithmes
de gestion optimale.
Ces outils se rattachent à théories de disciplines variées (Automatique, Optimisation, Recheche Opérationnelle),
qui ont chacune leur spécificité tout en se recouvrant partiellement.
%
Il est donc difficile, pour la personne ``non initiée'', de saisir les principales caractéristiques
de chaque approche pour pouvoir les comparer et finalement trouver
quelles méthodes sont plus adaptées à un problème donné.
%
Nous proposons ici, sur un exemple simple de système photovoltaïque-stockage, de
comparer différentes méthodes en soulignant en particulier les investissements
en temps à prévoir :
temps pour la compréhension du cadre théorique,
temps pour la modélisation du problème dans ce cadre,
temps pour l'implémentation numérique et la validation des résultats.
Nous soulignons également quelques pièges typiques, comme l'optimisation
déterministe anticipative dans un contexte stochastique.
}\\

\textbf{\textit{Gestion d'énergie, Optimisation dynamique, Optimisation stochastique,
Commande prédictive, Programmation Dynamique}}

\fontsize{10pt}{10pt}\selectfont


\section{Introduction}

\subsection{Système photovoltaïque-stockage}
Pour illustrer la problématique, on considère un système photovoltaïque-stockage
pour l'autoconsommation d'un consommateur résidentiel.

L'objectif de pilotage énergétique est la minimisation de la facture
de l'énergie consommmée du réseau. Le dimensionnement du système (capa stockage, pv, réseau) est considéré comme fixe.

Ausgrid data :

\url{https://www.ausgrid.com.au/Common/About-us/Corporate-information/Data-to-share/Solar-home-electricity-data.aspx}

\url{https://github.com/pierre-haessig/ausgrid-solar-data}


\section{Approches d'optimisation}




TODO: cite thèse geeps voiture

\subsection{Programmation dynamique}
bon cadre théorique,

pb à l'implémentation

\cite{Haessig:2013:ESPy}

\subsection{Optimisation déterministe anticipative}
le piège

\cite{Rigo-Mariani:2014:SGE}

poster µgrid powertech 2015 ?

\subsection{MPC déterministe}
le classique

\subsection{MPC stochastique et robuste}

une amélioration du MPC déterministe.

\subsection{Commande prédictive non-linéaire}

Optimica JModelica.org \cite{Akesson:2010:CCE}

sareni sge 2014 \cite{Rigo-Mariani:2014:SGE} : optim linéaire puis reprojection.

\begin{figure}[!ht]
	\begin{center}
		\includegraphics[width=0.6\columnwidth]{figures/wind_storage.pdf}
	\end{center}
	
	\caption{Titre de la figure (Style ‘Légende’)}
	\label{fig_1}
\end{figure}

\section{Conclusions}

Rappeler les principaux résultats marquants et originaux du travail. Le cas échéant, proposer des perspectives au travail présenté.

Perspectives : structure de décision dans un contexte multi-agent (pas abordé ici
car focus sur optimisation d'un système individuel) : centralisé, distribué,
mécanisme de prix, multi-agent...


\section{Remerciements}

Cette partie (facultative) doit être placée entre la conclusion et les références.


\bibliographystyle{IEEEtran}
\bibliography{00_References}


\end{document}

